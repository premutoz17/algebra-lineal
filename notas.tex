\documentclass{book}
\usepackage[utf8]{inputenc}
\usepackage[spanish]{babel}
\usepackage[mathscr,mathcal]{euscript}
\usepackage{amssymb}
\usepackage{amsthm}
\usepackage{amsmath}
\usepackage{shapepar}
\usepackage{latexsym}
\usepackage{graphicx}
\usepackage{color}
\usepackage{shapepar}
\newtheorem{theorem}{Teorema}
\newtheorem{proposition}{Proposición}
\newtheorem{definition}{Definición}
\newtheorem{axiom}{Axioma}
\newtheorem{corollary}{Corolario}
\newtheorem{lemma}[theorem]{Lemma}
\newtheorem*{remark}{Remark}

\title{Notas de Álgebra Lineal}
\author{Carlos Francisco Flores Galicia.}
\date{}

\begin{document}

\maketitle

\chapter{Espacios vectoriales}
\subsection{Espacios vectoriales}
\subsection{Subespacios vectoriales}
\subsection{Combinaciones lineales}

\begin{definition}
Sea $V$ un espacio vectorial y $S \subseteq V$, $S \neq \emptyset$. Se dice que un vector $v \in V$ es combinación lineal de elementos de $S$, si existe un conjunto finito $\{s_1,s_2,...,s_n\}\subseteq S$ y escalares $\lambda_1,\lambda_2,...\lambda_n \in K$ tales que $v=\lambda_1 s_1+\lambda_2 s_2+...+\lambda_n s_n$. Se dice también que $v$ es combinación lineal de $\{s_1,s_2,...,s_n\}$.
\end{definition}

\begin{definition}
Sea $V$ un espacio vectorial y $S \subseteq V$, $S \neq \emptyset$. Definimos al conjunto generado por elementos de $S$ como

\begin{equation}
\langle S \rangle = \{\lambda_1 s_1+\lambda_2 s_2+...+\lambda_n s_n : \lambda_1,\lambda_2,...\lambda_n \in K \}
\end{equation}

Esto es, el conjunto generado por $S$ es el conjunto de todas las combinaciones lineales de los elementos de $S$.
\end{definition}

\begin{definition}
$\langle \emptyset \rangle=\{0_k\}$
\end{definition}

\begin{theorem}
Sea $V$ un espacio vectorial y $S \subseteq V$, $S \neq \emptyset$, entonces $\langle S \rangle \leq V$ y $\langle S \rangle$ es el subespacio de $V$ más pequeño que contiene a $S$ (es decir, que $\langle S \rangle$ es un subconjunto de todos los subespacios de $V$ que contienen a $S$).
\end{theorem}
\begin{proof}
Probemos primero que $\langle S \rangle \leq V$. 
\end{proof}

\end{document}
